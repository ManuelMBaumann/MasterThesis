\chapter{Summary and Future research}
\section{Overview and main results}
In this project, we have applied the nonlinear MOR methods POD and DEIM to the optimal control of Burgers' equation. Therefore, the optimal control problem \eqref{minJ}-\eqref{Burgers2} has been discretized in space using a finite element approach and integrated in time using an implicit Euler scheme. The resulting discretized equations are an implicitly constrained optimization problem that has been solved using the three different optimization algorithms Newton-type method, BFGS and SPG. The main theoretical contribution of this work is the derivation of the adjoints equation in Algorithm \ref{alg:Adj1} and Algorithm \ref{alg:Adj2}and its application to the full-order and the POD-DEIM reduced Burgers' model in the Algorithms \ref{alg:Adj1_Burgers}-\ref{alg:Adj2_Burgers} and the Algorithms \ref{alg:Adj1_redBurgers}-\ref{alg:Adj2_redBurgers}, respectively. The discretized optimal control problem for Burgers' equation has been implemented in \textsc{Matlab} and the three different optimization algorithms have been tested with the full-order model, the POD-reduced model and the POD-DEIM-reduced model.

One main goal of this thesis was to apply DEIM to a POD-reduced model of Burgers' equation and compare the results of both reduced models with respect to accuracy of the dynamical behavior and computational speedup. The theoretical derivations in Section \ref{Deim_chap} have shown that the POD-DEIM reduced Burgers' model is completely independent of the size of the full-order model $N$ whereas in the case of a purely POD-reduced model, the nonlinear term of Burgers' equation still depends on $N$. The numerical tests in Section \ref{sect2_numTests} have shown that the larger the original dimension $N$ is, the more speedup we obtain from DEIM. For $N=800$ and a viscosity parameter of $\nu = 0.0001$ we have shown that the speedup of POD-DEIM is about $52$ times the full-order model while a POD-reduced model only leads to a speedup of $16$. Further numerical tests have proven the independence of the POD-DEIM model of $N$.

In Section \ref{Opt_chap} and \ref{chap4}, the DEIM method has been used for the optimal control of Burgers' equation. In this thesis, we have compared the optimal control of Burgers' equation using the three different optimization algorithms BFGS, SPG and a Newton-type method. All of them have been applied to the optimal control of the full-order model as well as a POD and a POD-DEIM reduced model. Numerical tests have been shown that the optimal state of the reduced models is close to the optimal state of the full-order model measured in the $L_2$-norm. At the same time we have seen for $\nu = 0.0001$ and $N = 800$ that a computational speedup of more than $100$ times for the POD-DEIM reduced optimal control is possible whereas the POD-reduced model only let to a speedup of $\sim 80$. Additionally, the SPG method has been used to introduce so-called bound constraints on the control of the full-order and the respective reduced models. Numerical tests in Section \ref{numTests} have shown that also for this case, the optimization of the POD-DEIM reduced model leads to a significant speedup and an optimal state that is close to the optimal state of the full-order model.
\section{Outlook on future research questions}
At this point, we would like to present some directions for future research that might build up on this thesis work.
\subsection*{A priori dimension reduction of the control variable}
In Section \ref{smallu_sec}, we have derived a POD-DEIM reduced optimal control problem that uses only a low-dimensional control in order to drive the solution of Burgers' equation into the desired state. This has been done by defining discrete control points at which the control is allowed to be different from zero. Numerical tests in Section \ref{numTests} have shown that this approach leads to a tremendous reduction in computational time even for the optimization of the full-order Burgers' model. It has been shown as well that the choice of the position of the control points is crucial and requires some \textit{physical inside} of the considered optimization problem. Since it is in general not clear which positions are optimal for the control points, it would be desirable to develop an algorithmic approach that reduces the dimension of the control and, hence, the dimension of the optimization problem and at the same time leads to an optimal state close to the desired state. A different and more general approach for the dimension reduction of the control variable might be to use the POD basis of the state variable. For the dimension reduction of the state variable we have used an approximation of the form,
\begin{align*}
\mathbf{y}(t) \approx \Phi_\ell \mathbf{\tilde y}(t),
\end{align*}
where $\Phi_\ell = [\varphi_1,...,\varphi_\ell]$ consists columnwise of the POD basis and, thus, the state variable can be expressed as $\mathbf{y}(t) \approx \sum_{i=1}^\ell \varphi_i \tilde{y}_i(t)$. In a future work, one might consider a similar approach also for the control variable,
\begin{align}
\label{lowu}
\mathbf{u}(t) \approx \Phi_\ell \mathbf{\tilde u}(t) = \sum_{i=1}^\ell \varphi_i \tilde{u}_i(t),
\end{align}
where $\mathbf{\tilde u}(t) = [\tilde{u}_1,...,\tilde{u}_\ell]^T$ is the reduced control. Note that we suggest to use the same basis $\{\varphi_i\}_{i=1}^\ell$ for the low-dimension expression of the control variable. In Section \ref{MOR_chap} we have argued that the POD basis captures well the dynamical behavior of the state variable. Therefore, the quality of the approximation \eqref{lowu} might be of interest in future research.
\subsection*{Optimal control of Burgers' equation in 2D/3D}
In order to obtain a larger factor for the computational speedup of DEIM, the work of \cite{CS10,SN13} has shown that a problem defined for the physical domain $\Omega \subset \mathbb{R}^{d}$, for $d = {2,3}$, shows in general more potential for model order reduction. In \cite{CS10}, the authors present the application of DEIM to the numerical simulation of a two-dimensional model for miscible fingering in porous media. Therein, it is presented that the POD-DEIM reduced model leads to a reduction of the computational time by a factor of $\mathcal{O}(1000)$. Since it has already been shown in \cite{GU12,RT07} that the optimal control for $2$ or $3$-dimensional models is possible, a further improvement of this work would be to extend the optimal control of Burgers' equation to higher dimensions and evaluate the computational gain of POD-DEIM. It is expected that this leads to a larger speedup for the reduced model.
\subsection*{Educated choice of the reduction dimensions $\ell$ and $m$}
During the numerical tests in Chapter \ref{chap4}, we have seen that it is in general not trivial to choose suitable values for the reduced dimensions $\ell$ and $m$. Especially when the SPG method has been used for the solution of the optimal control problem, we have seen in Section \ref{numTests} that it might happen that more function and gradient evaluations for the POD-DEIM reduced model are necessary which has a bad influence on the computational speedup even though a single optimization iteration is much faster. For the test calculations presented in Table \ref{time_messure3}, we have not been able to choose the DEIM dimension in such a way that this behavior does not appear. It is therefore desirable to evaluate the choice of $\ell$ and $m$ in more detail and derive an a priori estimate for both dimensions such that the convergence behavior of the SPG method is optimal.
\subsection*{Development of a reduced model for optimal flow control}
Burgers' equation is an important model equation in the field of computational fluid dynamics (CFD) because the structure of the nonlinear term is similar to the nonlinearity in the Navier-Stokes equations. Since this thesis has proven that POD-DEIM leads to a good approximation of the dynamical behavior of Burgers' equation, one might conclude that the same mathematical methods can be applied in order to derive a reduced model of the Navier-Stokes equations. In \cite{MS07,RT07}, the Navier-Stokes equations have been considered for the optimal control of two-dimensional flow. The implementation of MOR techniques to the software used in \cite{MS07,RT07} might be a major step in order to reduce the huge computational work that is required in CFD applications. The theoretical considerations for Burgers' equation presented in this thesis might be very helpful in order to derive a POD-DEIM reduced model for the Navier-Stokes equations due to the similar nonlinear terms of both models.
%\subsection*{FEM-POD}
